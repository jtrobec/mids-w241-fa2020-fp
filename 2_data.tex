% Options for packages loaded elsewhere
\PassOptionsToPackage{unicode}{hyperref}
\PassOptionsToPackage{hyphens}{url}
%
\documentclass[
]{article}
\usepackage{lmodern}
\usepackage{amssymb,amsmath}
\usepackage{ifxetex,ifluatex}
\ifnum 0\ifxetex 1\fi\ifluatex 1\fi=0 % if pdftex
  \usepackage[T1]{fontenc}
  \usepackage[utf8]{inputenc}
  \usepackage{textcomp} % provide euro and other symbols
\else % if luatex or xetex
  \usepackage{unicode-math}
  \defaultfontfeatures{Scale=MatchLowercase}
  \defaultfontfeatures[\rmfamily]{Ligatures=TeX,Scale=1}
\fi
% Use upquote if available, for straight quotes in verbatim environments
\IfFileExists{upquote.sty}{\usepackage{upquote}}{}
\IfFileExists{microtype.sty}{% use microtype if available
  \usepackage[]{microtype}
  \UseMicrotypeSet[protrusion]{basicmath} % disable protrusion for tt fonts
}{}
\makeatletter
\@ifundefined{KOMAClassName}{% if non-KOMA class
  \IfFileExists{parskip.sty}{%
    \usepackage{parskip}
  }{% else
    \setlength{\parindent}{0pt}
    \setlength{\parskip}{6pt plus 2pt minus 1pt}}
}{% if KOMA class
  \KOMAoptions{parskip=half}}
\makeatother
\usepackage{xcolor}
\IfFileExists{xurl.sty}{\usepackage{xurl}}{} % add URL line breaks if available
\IfFileExists{bookmark.sty}{\usepackage{bookmark}}{\usepackage{hyperref}}
\hypersetup{
  pdftitle={intro},
  pdfauthor={Bill Chung, Justin Trobec, Nobu Yamaguchi},
  hidelinks,
  pdfcreator={LaTeX via pandoc}}
\urlstyle{same} % disable monospaced font for URLs
\usepackage[margin=1in]{geometry}
\usepackage{longtable,booktabs}
% Correct order of tables after \paragraph or \subparagraph
\usepackage{etoolbox}
\makeatletter
\patchcmd\longtable{\par}{\if@noskipsec\mbox{}\fi\par}{}{}
\makeatother
% Allow footnotes in longtable head/foot
\IfFileExists{footnotehyper.sty}{\usepackage{footnotehyper}}{\usepackage{footnote}}
\makesavenoteenv{longtable}
\usepackage{graphicx,grffile}
\makeatletter
\def\maxwidth{\ifdim\Gin@nat@width>\linewidth\linewidth\else\Gin@nat@width\fi}
\def\maxheight{\ifdim\Gin@nat@height>\textheight\textheight\else\Gin@nat@height\fi}
\makeatother
% Scale images if necessary, so that they will not overflow the page
% margins by default, and it is still possible to overwrite the defaults
% using explicit options in \includegraphics[width, height, ...]{}
\setkeys{Gin}{width=\maxwidth,height=\maxheight,keepaspectratio}
% Set default figure placement to htbp
\makeatletter
\def\fps@figure{htbp}
\makeatother
\setlength{\emergencystretch}{3em} % prevent overfull lines
\providecommand{\tightlist}{%
  \setlength{\itemsep}{0pt}\setlength{\parskip}{0pt}}
\setcounter{secnumdepth}{-\maxdimen} % remove section numbering

\title{intro}
\author{Bill Chung, Justin Trobec, Nobu Yamaguchi}
\date{11/21/2020}

\begin{document}
\maketitle

\hypertarget{sata}{%
\section{2. Sata}\label{sata}}

\hypertarget{description-of-data}{%
\subsubsection{Description of data}\label{description-of-data}}

\hypertarget{apple}{%
\subsubsection{Apple}\label{apple}}

Daily pictures of the apples were taken and the raw data from all three
blocks are stroed in xx. The detaild python code for converting apples
pictures to data are included in xx and qualitative description of the
flow are descripted in here.

\texttt{Treatment} was ..

This was used as pre-experiment data to develop the process of the
converting picture to xx.

The initital condition was used to quantify the changes in apple skin
colors.

\hypertarget{banana}{%
\subsubsection{Banana}\label{banana}}

Daily pictures of banana were taken..

The picture (xx) that were read using Opencv library were first
decomposed into Hue, Saturation and Value (HSA) and the range of Hue,
Saturation and Value that will be used to filter banana image only were
determined using \texttt{getEdage.py} and the information found at
\textbf{\url{http://colorizer.org}}

After first filtering images, \texttt{img}, based on predifined HSA, the
filtered images, \texttt{mask}, were used to then conver the background
including the apples (see figure) into black and saved as
\texttt{imgResult}. Then, contours of the banana were extracted together
with other noise in the figure, then only the regions with closed
contour with area greater than predified value were extracted and saved
for the analysis.

The initital condition was used to quantify the changes in banana skin
colors.

\hypertarget{avocado}{%
\subsubsection{Avocado}\label{avocado}}

Unlike apples and avocad, the observation of avocados that received
treatments and control was made once on day xx by slicing avocados in
1/2 and comparing the fraction of avocado region that turned
brown\ldots{}

\hypertarget{converting-picture-to-data}{%
\subsection{Converting picture to
data}\label{converting-picture-to-data}}

\hypertarget{data-from-expereiment-4}{%
\subsection{Data from expereiment 4}\label{data-from-expereiment-4}}

\begin{longtable}[]{@{}cl@{}}
\toprule
Names & description\tabularnewline
\midrule
\endhead
Block & name of the block, \texttt{B}, \texttt{J} and
\texttt{N}\tabularnewline
id & two digit id which identifies individual subject\tabularnewline
Yi & 0 if it was control and 1 if it was treatment\tabularnewline
Pt & Changes in subject condition observed in day t with respect to day
1\tabularnewline
\bottomrule
\end{longtable}

The chagnes in subject condition is measured in the following procedure.
The picture of apple cropped based on its contour in the picture were
converted to 256 by 256 matrix in grey scale. This matrix is then
converted to a vector and normalized.

\(\vec{v}_1\) represent the condition of apple in day 1.

The picture of the apple from the subsequent days were also cropped
based on the procedure descriped in \texttt{method}, then stored in
\(\vec{v}_2 .. \vec{v}_{25}\).\\
The EX1 was ended on day 25 since the both the skin of the apples in
both the control and treatments remain unchanged compared to day 1.
Considering the time limit, the team decided to end the experiment on
day 25 and start EX2, EX3 and EX4.

Pt is defined as the following

\[Pt = \frac{\vec{v}_1 \cdot \vec{v}_t}{\vec{v}_1^T \cdot \vec{v}_1}\]

Since all the vectors are normalized, \(Pt\) will range between 0 and 1.
0 indicating The apple from day t looks completely different from day 0.
1 indicating that the condition did not change at all.

The plot of \(pt\) for all control and treatment for each blocks are
shown in the fiugre below.

\hypertarget{data-from-expereiment-4-1}{%
\subsection{Data from expereiment 4}\label{data-from-expereiment-4-1}}

\begin{longtable}[]{@{}cl@{}}
\toprule
Names & description\tabularnewline
\midrule
\endhead
Block & name of the block, \texttt{B}, \texttt{J} and
\texttt{N}\tabularnewline
id & two digit id which identifies individual subject\tabularnewline
Yi & 0 if it was control and 1 if it was treatment\tabularnewline
Cafe Noir & RGB(84,59,35)\tabularnewline
Pastel Brown & RGB(132,105,84)\tabularnewline
White Coffee & RGB(223,223,212)\tabularnewline
June Bud & RGB(165,215,77)\tabularnewline
Dark Lemon Lime & RGB(129,190,28)\tabularnewline
White & RGB(255,255,255)\tabularnewline
Black & RGB(0,0,0)\tabularnewline
\bottomrule
\end{longtable}

\href{https://www.schemecolor.com/green-with-brown-color-combination.php}{LINK}

\begin{quote}
Apply KNN to clsuter the pixles. Remove the ones that are clustered with
white and black Normalized the ones that are in other groups. Compare
the composition.
\end{quote}

\end{document}
